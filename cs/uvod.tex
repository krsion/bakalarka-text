\chapter*{Úvod}
\addcontentsline{toc}{chapter}{Úvod}

Meteorologické webové kamery představují významný zdroj informací o počasí, oblačnosti a dalších atmosférických jevech. 
Kromě přímého využití v meteorologii mohou tyto kamery sloužit i jako zdroj trénovacích dat pro neuronové sítě generující realistické mraky na obloze. 
Sběr dat z webových meteorologických kamer je pro výzkumníky neuronových sítí levnější než focení oblohy ručně.
Tyto neuronové sítě mají uplatnění například ve filmovém průmyslu, videohrách či simulacích. 
Pro dosažení vysoké úrovně realismu je důležitá správná kalibrace webových kamer, což zahrnuje určení parametrů kamery, jako jsou azimut, zenit, natočení a zorné pole. Tyto údaje však nemusejí být známé, proto existují metody pro jejich aproximaci.

Cílem této bakalářské práce je porovnat stávající přístupy k automatické kalibraci kamer, konkrétně metody \cite{Lalonde10} a \cite{deepcalib}. Následně je zamýšleno vylepšit první zmíněnou metodu integrací modelu oblohy Prague Sky Model \citep{Prague2021}.

Pro validaci navrženého řešení  bude využit dataset webových kamer od Českého hydrometeorologického ústavu (ČHMÚ), který poskytuje obrazová data z 98 webových kamer.