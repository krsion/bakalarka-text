\chapter*{Úvod}
\addcontentsline{toc}{chapter}{Úvod}

Meteorologické webové kamery představují významný zdroj informací o počasí, oblačnosti a dalších atmosférických jevech. Kromě přímého využití v meteorologii mohou tyto kamery sloužit i pro další účely, jako je renderování realistických mraků a oblohy v různých aplikacích, například ve filmovém průmyslu, videohrách či simulacích. Klíčovým faktorem pro dosažení vysoké úrovně realismu v takových aplikacích je správná kalibrace webových kamer, což zahrnuje určení parametrů kamery, jako jsou azimut, zenit, natočení a zorné pole. Tyto údaje však nemusejí být známé, proto existují metody pro jejich aproximaci.

Cílem této bakalářské práce je porovnat stávající přístupy k automatickému nalezení chybějících parametrů meteorologických kamer, konkrétně metody Lalonde08/10 a Hold-Geoffroy. Následně je zamýšleno vylepšit metodu Lalonde08/10 integrací modelu oblohy Prague Sky Model.

Pro validaci navrženého řešení  bude využit dataset webových kamer od Českého hydrometeorologického ústavu (ČHMÚ), který poskytuje obrazová data z 98 webových kamer.